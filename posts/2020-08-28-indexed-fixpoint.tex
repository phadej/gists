\section{Introduction}

I was lately thinking about fixed points, more or less.

A new version of 
\href{https://hackage.haskell.org/package/data-fix}{\texttt{data-fix}}
was released recently, and also corresponding version
of \href{https://hackage.haskell.org/package/recursion-schemes}{\texttt{recursion-schemes}}.

Also I wrote a \href{https://www.well-typed.com/blog/2020/06/fix-ing-regular-expressions/}{Fix-ing regular expressions} post,
about adding fixed points to regular expression.

This post is another exploration: \emph{Fixed points of Indexed functors}.
This is not novel idea at all,
but I'm positively surprised this works out quite nicely in modern GHC Haskell.
I define a |IxFix| type and illustrate it with three examples.

\emph{Note:} The
\href{https://hackage.haskell.org/package/multirec-0.7.9/docs/Generics-MultiRec-HFix.html}{|HFix| in \texttt{multirec} package} is the same as |IxFix|
in this post. I always forget about the existence of \texttt{multirec}.

In the following, the "modern GHC Haskell" is quite conservative,
only eight extensions:

\begin{code}
{-# LANGUAGE DataKinds #-}
{-# LANGUAGE GADTs #-}
{-# LANGUAGE PolyKinds #-}
{-# LANGUAGE RankNTypes #-}
{-# LANGUAGE ScopedTypeVariables #-}
{-# LANGUAGE TypeFamilies #-}
{-# LANGUAGE TypeOperators #-}
\end{code}

And this literate Haskell script is warning free, with \texttt{-Wall}
\begin{code}
{-# OPTIONS_GHC -Wall #-}
\end{code}

On this trip
\begin{code}
module IxFix where
\end{code}

we need a handful of imports

\begin{code}
-- Type should be added to Prelude
import Data.Kind (Type)

-- Few newtypes
import Data.Functor.Identity (Identity (..))
import Data.Functor.Compose (Compose (..))
import Data.Functor.Const (Const (..))

-- dependently typed programming!
import Data.Fin      (Fin)
import Data.Type.Nat
import Data.Vec.Lazy (Vec (..))

-- magic
import Data.Coerce (coerce)
\end{code}

Before we go further, let me remind you about ordinary fixed points,
as defined in 
\href{https://hackage.haskell.org/package/data-fix}{\texttt{data-fix}}
package.

\begin{code}
newtype Fix f = Fix { unFix :: f (Fix f) }

foldFix :: Functor f => (f a -> a) -> Fix f -> a
foldFix f = go where go = f . fmap go . unFix
\end{code}

Using |Fix| we can define recursive types using non-recursive \emph{base functors},
e.g. for a list we'd have

\begin{code}
data ListF a rec = NilF | ConsF a rec
\end{code}

We then use |foldFix| (or |cata| and other recursion schemes in \texttt{recursion-schemes}) to 
decouple "how we recurse" and "what we do at each step".
I won't try to convince you why this separation of concerns might be useful.

Instead I continue directly to the topic: define indexed fixed points.
Why we need them? Because |Fix| is not powerful enough to allow
working with |Vec| or polymorphically recursive types.

\begin{code}
-- hello dependently typed world.
data Vec (n :: Nat) (a :: Type) where
    VNil  :: Vec 'Z a
    (:::) :: a -> Vec n a -> Vec ('S n) a
\end{code}

\section{Fixed points of indexed functors}

Before talking about fixed points, we need to figure out what
are indexed functors. Recall a normal functor
is a thing of kind |Type -> Type|:

\begin{code}
class Functor f where
    fmap :: (a -> b) -> (f a -> f b)
\end{code}

Indexed version is the one with |Type| replaced with |k -> Type|,
for some index |k| (it is still a functor, but in different category).
We want morphisms to work for all indices,
and preserve them. Thus we define a commonly used type alias\footnote{I'm sorry that tilde |~>| and dash |->| arrows look so similar.}
\begin{code}
-- natural, or parametric, transformation
type f ~> g = forall (j :: k). f j -> g j
\end{code}
Using it we can define a |Functor| variant\footnote{
Note that |FFunctor| in \url{https://hackage.haskell.org/package/hkd-0.1/docs/Data-HKD.html} (which is defined with different names in other packages as well)
is of different kind.
|IxFunctor| in \url{https://hackage.haskell.org/package/indexed-0.1.3/docs/Data-Functor-Indexed.html}
is again different.
Sorry for proliferation of various functors.
And for confusing terminology. 
Dominic Orchard et al uses terms graded (|k -> Type|, this post) and parameterised (|k -> k -> Type|, \texttt{indexed}-package) in \url{https://arxiv.org/abs/2001.10274v2}.
There is no monad-name for \texttt{hkd}-package variant, as that cannot be made into monad-like thing.
}:
it looks almost the same.
\begin{code}
class IxFunctor (f :: (k -> Type) -> (k -> Type)) where
    ixmap :: (a ~> b) -> (f a ~> f b)
\end{code}

With |IxFunctor| in our toolbox, we can define an |IxFix|,
note how the definition is again almost the same as for unindexed |Fix| and |foldFix|:
\begin{code}
newtype IxFix f i = IxFix { unIxFix :: f (IxFix f) i }

foldIxFix :: IxFunctor f => (f g ~> g) -> IxFix f ~> g
foldIxFix alg = alg . ixmap (foldIxFix alg) . unIxFix
\end{code}

Does this work?
I hope that following examples will convince that |IxFix|
is usable (at least in theory).

\section{Example: length indexed lists, Vec}

The go to example of recursion schemes is a folding a list,
The go to example of dependent types is length indexed list, often called |Vec|.
I combine these traditions by defining |Vec| as an indexed fixed point:

\begin{code}
data VecF (a :: Type) rec (n :: Nat) where
    NilF  ::               VecF a rec 'Z
    ConsF :: a -> rec n -> VecF a rec ('S n)
\end{code}

|VecF| is an |IxFunctor|:

\begin{code}
instance IxFunctor (VecF a) where
    ixmap _ NilF         = NilF
    ixmap f (ConsF x xs) = ConsF x (f xs)
\end{code}

And we can define |Vec| as fixed point of |VecF|, with constructors:

\begin{code}
type Vec' a n = IxFix (VecF a) n

nil :: Vec' a 'Z
nil = IxFix NilF

cons :: a -> Vec' a n -> Vec' a ('S n)
cons x xs = IxFix (ConsF x xs)
\end{code}

Can we actually use it? Of course!
Lets define concatenation\footnote{You may wonder why function name is |append|, but operation is concatenation? This is similar to having |plus| for addition.}
|Vec' a n -> Vec' a m -> Vec' a (Plus n m)|.
We cannot use |foldIxFix| directly, as |Plus n m| is not the same
index as |n|, so we need to define an auxiliary |newtype|
to plumb the indices. Another way to think about these kind of |newtype|s,
is that they work around the lack of type-level anonymous functions in nowadays Haskell.

\begin{code}
newtype Appended m a n =
    Append { getAppended :: Vec' a m -> Vec' a (Plus n m) }
\end{code}

\begin{code}
append :: forall a n m. Vec' a n -> Vec' a m -> Vec' a (Plus n m)
append xs ys = getAppended (foldIxFix alg xs) ys where
    alg :: VecF a (Appended m a) j -> Appended m a j
    alg NilF          = Append id
    alg (ConsF x rec) = Append $ \zs -> cons x (getAppended rec zs)
\end{code}

We can also define a refold function, which doesn't mention
|IxFix| at all.

\begin{code}
ixrefold :: IxFunctor f => (f b ~> b) -> (a ~> f a) -> a ~> b
ixrefold f g = f . ixmap (ixrefold f g) . g
\end{code}

And then, using |ixrefold| we can define concatenation for |Vec| from
\href{https://hackage.haskell.org/package/vec}{\texttt{vec}} package,
which isn't defined using |IxFix|.
Here we need auxiliary |newtype|s as well.

\begin{code}
newtype Swapped f a b =
    Swap { getSwapped :: f b a }
newtype Appended2 m a n =
    Append2 { getAppended2  :: Vec m a -> Vec (Plus n m) a }

append2 :: forall a n m. Vec n a -> Vec m a -> Vec (Plus n m) a
append2 xs ys = getAppended2 (ixrefold f g (Swap xs)) ys where
    -- same as alg in 'append'
    f :: VecF a (Appended2 m a) j -> Appended2 m a j
    f NilF          = Append2 id
    f (ConsF z rec) = Append2 $ \zs -> z :::  (getAppended2 rec zs)

    -- 'project'
    g :: Swapped Vec a j -> VecF a (Swapped Vec a) j
    g (Swap VNil)       = NilF
    g (Swap (z ::: zs)) = ConsF z (Swap zs)
\end{code}

You may note that one can implement |append| as induction over length,
that's how |vec| implements them.
Theoretically it is not right, and |IxFix| formulation highlights it:

\begin{code}
append3 :: forall a n m. SNatI n
        => Vec' a n -> Vec' a m -> Vec' a (Plus n m)
append3 xs ys = getAppended3 (induction caseZ caseS) xs where
    caseZ :: Appended3 m a 'Z
    caseZ = Append3 (\_ -> ys)

    caseS :: Appended3 m a p -> Appended3 m a ('S p)
    caseS rec = Append3 $ \(IxFix (ConsF z zs)) ->
        cons z (getAppended3 rec zs)

-- Note: this is different than Appended!
newtype Appended3 m a n =
    Append3 { getAppended3 :: Vec' a n -> Vec' a (Plus n m) }
\end{code}

Here we \emph{pattern match} on |IxFix| value.
If we want to treat it as least fixed point, the only valid elimination
is to use |foldIxFix|!

However, the induction over length is the right approach if |Vec| is defined
as a data or type family:

\begin{code}
type family VecFam (a :: Type) (n :: Nat) :: Type where
    VecFam a 'Z     = ()
    VecFam a ('S n) = (a, VecFam a n)
\end{code}

Whether you want to have data or type-family or GADT depends on the application.
(Even in Agda or Coq). Family variant doesn't intristically know its length,
which is sometimes a blessing, sometimes a curse.
For what it's worth, \texttt{vec} package provides both variants,
with almost the same module interface.

\section{Example: Polymorphically recursive type}

The |IxFix| can also be used to define polymorphically recursive types like

\begin{code}
data Nested a = a :<: (Nested [a]) | Epsilon
infixr 5 :<:

nested :: Nested Int
nested = 1 :<: [2,3,4] :<: [[5,6],[7],[8,9]] :<: Epsilon
\end{code}

A length function defined over this datatype will be polymorphically recursive,
as the type of the argument changes from Nested a to Nested [a] in the recursive call:

\begin{code}
-- >>> nestedLength nested
-- 3
nestedLength :: Nested a -> Int
nestedLength Epsilon    = 0
nestedLength (_ :<: xs) = 1 + nestedLength xs
\end{code}

We cannot represent |Nested| as |Fix| of some functor,
and we can not use \texttt{recursion-schemes} either.
However, we can redefine |Nested| as indexed fixed point.

An important observation is that we (often or always?) use polymorphic recursion
as a solution to the lack indexed types.
My favorite example is de Bruijn indices for well-scoped terms.
Compare
\begin{code}
data Expr1 a
    = Var1 a
    | App1 (Expr1 a) (Expr1 a)
    | Abs1 (Expr1 (Maybe a))
\end{code}
and
\begin{code}
data Expr2 a n
    = Free2 a                -- split free and bound variables
    | Bound2 (Fin n)
    | App2 (Expr2 a n) (Expr2 a n)
    | Abs2 (Expr2 a ('S n))  -- extend bound context by one
\end{code}
Which one is \emph{simpler} is a really good discussion,
but for another time.

In |Nested| example the single argument is also used for two
purposes: the type of an base element (|Int|) and container type
(starts with |Identity| and increases with extra list layer).

One approach is just use |Nat| index and have a type family%
\footnote{Here one starts to wish that GHC had unsaturated type families, so we wouldn't need to use newtypes...}
\begin{code}
type family Container (n :: Nat) :: Type -> Type where
    Container 'Z     = Identity
    Container ('S n) = Compose [] (Container n)
\end{code}

or

\begin{code}
data NestedF a rec f
    = f a :<<: rec (Compose [] f)
    | EpsilonF

instance IxFunctor (NestedF a) where
    ixmap _ EpsilonF    = EpsilonF
    ixmap f (x :<<: xs) = x :<<: f xs
\end{code}

We can convert from |Nested a| to |IxFix (NestedF a) Identity| and back.
We use |coerce| to help with |newtype| plumbing.

\begin{code}
convert :: Nested a -> IxFix (NestedF a) Identity
convert = aux . coerce where
    aux :: Nested (f a) -> IxFix (NestedF a) f
    aux Epsilon    = IxFix EpsilonF
    aux (x :<: xs) = IxFix (x :<<: aux (coerce xs))

-- back left as an exercise
\end{code}

And then we can write |nestedLength| as a fold.

\begin{code}
-- >>> nestedLength2 (convert nested)
-- 3
nestedLength2 :: IxFix (NestedF a) f -> Int
nestedLength2 = getConst . foldIxFix alg where
    alg :: NestedF a (Const Int) ~> Const Int
    alg EpsilonF         = Const 0
    alg (_ :<<: Const n) = Const (n + 1)
\end{code}

\section{Non-example: ListF}

In the introduction I mentioned an ordinary list, which is a fixed point

\begin{align*}
  \mathsf{List} \coloneqq \lambda (A : \mathsf{Type}).\; \mu (r : \mathsf{Type}).\; 1 + A \times r
\end{align*}

where I use $\mu (r : X).\, F\, r$ notation to represent least fixed points:
$\mu (r : X). F\, r \cong F\, (\mu (r : X). F\, r)$.
Note that we first introduce a type parameter $A$ with $\lambda$, and then make
a fixed point with $\mu$.

We can define an ordirinary list using |IxFix|,
by taking a fixed point of a |Type -> Type| thing,
i.e. first $\mu$, and then $\lambda$.

\begin{align*}
  \mathsf{List}_1 \coloneqq \mu (r : \mathsf{Type} \to \mathsf{Type}).\; \lambda (A : \mathsf{Type}).\; 1 + A \times r\,A
\end{align*}

\begin{code}
data ListF1 rec a = NilF1 | ConsF1 a (rec a)
type List1 = IxFix ListF1

fromList1 :: [a] -> List1 a
fromList1 []     = IxFix NilF1
fromList1 (x:xs) = IxFix (ConsF1 x (fromList1 xs))
\end{code}

Compare to Agda code:

\begin{code}
-- parameter
data List (A : Set) : Set where
    nil  : List A
    cons : A -> List A -> List A

-- index
data List : Set -> Set where
    nil :  (A : Set) -> List A
    cons : (A : Set) -> A -> List A -> List A
\end{code}

These types are subtly different.
See \url{https://stackoverflow.com/questions/24600256/difference-between-type-parameters-and-indices}.

This gives a hint why |Agda| people define |Vec (A : Set) : Nat -> Set|,
i.e. length as the last parameter: because you have to do that way.
And Haskellers (usually) define as |Vec (n :: Nat) (a :: Type)|,
because then |Vec| can be given |Functor| etc instances.
In other words, machinery in both languages forces an order of
of type arguments.

Finally, we can write parametric version of |List| using
|IxFix| too. We just use a dummy, boring index.

\begin{align*}
  \mathsf{List}_2 \coloneqq
  \lambda (A : \mathsf{Type}).\;
  \mu (r : 1 \to \mathsf{Type}).\;
  \lambda (x : 1).\;
  1 + A \times r\,x
\end{align*}

\begin{code}
data ListF2 a rec (unused :: ()) = NilF2 | ConsF2 a (rec unused)
type List2 a = IxFix (ListF2 a) '()

fromList2 :: [a] -> List2 a
fromList2 []     = IxFix NilF2
fromList2 (x:xs) = IxFix (ConsF2 x (fromList2 xs))
\end{code}

|IxFix| is more general than |Fix|, but if you don't need
an extra power, maybe you shouldn't use it.

Do we need something even more powerful than |IxFix|?
I don't think so. If we need more (dependent) indices, we can pack them all into
a single index by tupling (or $\sum$-mming) them.

\section{Conclusion}

We have seen |IxFix|, fixed point of indexed functor.
I honestly do not think that you should start looking in your code base whether you can use it.
I suspect it is more useful as thinking and experimentation tool.
It is an interesting gadget.
