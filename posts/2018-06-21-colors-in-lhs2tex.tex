\documentclass{article}
%include polycode.fmt

\usepackage{hyperref}
\hypersetup{pdfborder={0 0 0}}

% in lhs2TeX.style there are
%
% \newcommand{\Conid}[1]{{\mathit #1}}
% \newcommand{\Varid}[1]{{\mathit #1}}
% \newcommand{\anonymous}{\_}
%
% We can renew these

\usepackage{xcolor}
\definecolor{darkred}{rgb}{.5,0,0}
\definecolor{darkgreen}{rgb}{0,0.5,0}
\definecolor{darkblue}{rgb}{0,0,.5}
\definecolor{color4}{rgb}{0,.4,.4}
\definecolor{color5}{rgb}{.4,.4,0}

\renewcommand{\Conid}[1]{{\color{darkblue}\mathit #1}}

% however types and constructors look the same, we can differentiate them though

%format Foo = "{\color{darkred}\mathit Foo}"
%format MkFoo = "{\color{darkgreen}\mathit Foo}"

% Note how I "cheat" making MkFoo render as Foo!

% We can also highlight operators
%format + = "\mathbin{\color{color4}+}"

% or symbols (note I also make thing look prettier)"
%format plusFoo = "{\color{color5}\mathit plus_{Foo}}"
%format ColorsInLhs2TeX = "{\text{Colors in lhs2\TeX}}"

\begin{document}

An example of colorful lhs2\TeX\ file.
See the source at \url{https://github.com/phadej/gists/blob/master/posts/2018-06-21-colors-in-lhs2tex.tex}%
\footnote{It's named tex to trick \emph{Pandoc} in my blog setup},
and the result PDF at \url{https://github.com/phadej/gists/blob/master/pdf/ColorsInLhs2TeX.pdf}

\begin{code}
module ColorsInLhs2TeX where

newtype Foo = MkFoo Int

plusFoo :: Foo -> Foo -> Int
plusFoo (MkFoo n) (MkFoo m) = n + m
\end{code}
\end{document}
