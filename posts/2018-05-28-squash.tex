\documentclass[10pt,a4paper,pdftex]{article}
\usepackage[utf8]{inputenc}
\usepackage[T1]{fontenc}
%\usepackage{fullpage}
\usepackage{centernot}
\usepackage{amsmath}
\usepackage{amssymb,amsthm}
\usepackage{mathrsfs} % mathscr
\usepackage{mathtools} % coloneq
\usepackage[makeroom]{cancel} % cancel
\usepackage{color}
\usepackage[a4paper,inner=3cm,outer=3cm,top=3cm,bottom=3cm,marginparsep=5mm,marginparwidth=3cm]{geometry}
\usepackage{microtype}
\usepackage{float}
\usepackage[english]{babel}

% footnotes
\interfootnotelinepenalty=10000

% agdaid
\usepackage{xstring}
\newcommand{\agdaid}[1]{\mathit{#1}}

% TODO:
\usepackage{prftree}
% ftp://ftp.funet.fi/pub/TeX/CTAN/macros/latex/contrib/prftree/prftreedoc.pdf
\setlength{\prfinterspace}{1.5em}
\setlength{\prfrulenameskip}{0.3em}
\setlength{\prflinethickness}{.4pt}

% Verbatim
\usepackage{fancyvrb}

% Landscape
\usepackage{lscape}

% Colors
\usepackage{xcolor}
\definecolor{darkgreen}{rgb}{0,.5,0}
\definecolor{darkblue}{rgb}{0,0,.5}

% Fonts
\usepackage{mathpazo}
\usepackage{helvet}
%\usepackage{courier}
\usepackage{inconsolata}
\usepackage{microtype}

% graphics
\usepackage{graphics}
\DeclareGraphicsRule{*}{mps}{*}{}

% tables
\usepackage{booktabs}

% braces
\usepackage{stmaryrd}

% bibliography
\usepackage[square,sort&compress,numbers]{natbib}
\bibliographystyle{acm}
\renewcommand*{\bibfont}{\footnotesize}

% lhs2TeX
% \usepackage{poly}

% enumerations
\usepackage{enumitem}

% stuff
\usepackage{calc}

% Unicode
\usepackage{newunicodechar}
\newunicodechar{ℕ}{\ensuremath{\mathbb{N}}}
\newunicodechar{∀}{\ensuremath{\mathbb{\forall}}}
\newunicodechar{′}{'}
\newunicodechar{∷}{::}
\newunicodechar{≡}{\ensuremath{\equiv}}
\newunicodechar{→}{\ensuremath{\to}}
\newunicodechar{⇒}{\ensuremath{\Rightarrow}}
\newunicodechar{≤}{\ensuremath{\le}}
\newunicodechar{⊔}{\ensuremath{\sqcup}}
\newunicodechar{⊥}{\ensuremath{\bot}}
\newunicodechar{₁}{\ensuremath{{}_1}}
\newunicodechar{₂}{\ensuremath{{}_2}}
\newunicodechar{λ}{\ensuremath{\lambda}}
\newunicodechar{ℓ}{\ensuremath{\ell}}
\newunicodechar{Γ}{\ensuremath{\Gamma}}
\newunicodechar{Δ}{\ensuremath{\Delta}}
\newunicodechar{∈}{\ensuremath{\in}}
\newunicodechar{∘}{\ensuremath{\circ}}
\newunicodechar{ε}{\ensuremath{\varepsilon}}
\newunicodechar{τ}{\ensuremath{\tau}}
\newunicodechar{σ}{\ensuremath{\sigma}}
\newunicodechar{Σ}{\ensuremath{\Sigma}}
\newunicodechar{⊤}{\ensuremath{\top}}
\newunicodechar{⋆}{\ensuremath{\star}}

% finnish unicode :P
\newunicodechar{Ö}{\"O}
\newunicodechar{Ä}{\"A}

% References
\usepackage[hyphens]{url}
\usepackage[pdftex]{hyperref}
\hypersetup{pdfborder={0 0 0}}
\hypersetup{bookmarksnumbered=true}
% The following line suggests the PDF reader that it should show the
% first level of bookmarks opened in the hierarchical bookmark view.
\hypersetup{bookmarksopen=true,bookmarksopenlevel=1}
% Hyperref can also set up the PDF metadata fields. These are
% set a bit later on, after the thesis setup.
\hypersetup{colorlinks=true,linkcolor=darkgreen,urlcolor=darkblue}
\usepackage[capitalize,nameinlink,noabbrev]{cleveref}

% Theorems
% http://en.wikibooks.org/wiki/LaTeX/Theorems

% http://tex.stackexchange.com/questions/46258/how-to-get-correct-autoref-for-theorems
%\theoremstyle{definition}
%\newtheorem{definition}{Definition}[chapter]
%\newtheorem{example}[definition]{Example}
%\newtheorem{remark}[definition]{Remark}

%\newcommand{\definitionautorefname}{Definition}
%\newcommand{\exampleautorefname}{Definition}

% Currently the English versions are used for the PDF file metadata
% Set the PDF title
%\hypersetup{pdftitle={\TITLE\ \SUBTITLE}}
% Set the PDF author
%\hypersetup{pdfauthor={\AUTHOR}}
% Set the PDF keywords
%\hypersetup{pdfkeywords={\KEYWORDS}}
% Set the PDF subject
%\hypersetup{pdfsubject={Master's Thesis}}

% Geometry

% Line height
%\setlength{\baselineskip}{20mm}
% \setlength{\linespread}{1.5}

% paragraph style
\setlength{\parindent}{0em}
\setlength{\parskip}{0.3em}

% pandoc
\providecommand{\tightlist}{%
  \setlength{\itemsep}{0pt}\setlength{\parskip}{0pt}}

%include polycode.fmt
%include forall.fmt
%format <> = "\diamond"
\begin{document}
|Squashed c x| let a library writer provide |x| in "|c|-irrelevant" way to a
library user.

%if 0
\begin{code}
{-# LANGUAGE ConstraintKinds #-}
{-# LANGUAGE DeriveFunctor #-}
{-# LANGUAGE FlexibleInstances #-}
{-# LANGUAGE RankNTypes #-}
module Squash where
import Control.Applicative (liftA2)
import Control.Monad (void, liftM, liftM2)
import Data.Monoid (Sum (..))
import Data.Semigroup (Semigroup (..))
import Data.Tree (Tree (..))
import Data.Set (Set)
import qualified Data.Set as Set
\end{code}

The definition is simple:
%endif
\begin{code}
newtype Squashed c x = Squash
    { getSquashed :: forall r. c r => (x -> r) -> r }
\end{code}

|Squashed| is almost like
\href{https://hackage.haskell.org/package/transformers-0.5.5.0/docs/Control-Monad-Trans-Cont.html\#t:ContT}{|Cont|}%
\footnote{\url{https://hackage.haskell.org/package/transformers-0.5.5.0/docs/Control-Monad-Trans-Cont.html\#t:ContT}} or
\href{http://hackage.haskell.org/package/kan-extensions-5.1/docs/Control-Monad-Codensity.html\#t:Codensity}{|Codensity|}
\footnote{\url{http://hackage.haskell.org/package/kan-extensions-5.1/docs/Control-Monad-Codensity.html\#t:Codensity}},
so |Squashed| is a |Monad|:

\begin{code}
instance Monad (Squashed c) where
    return x  = Squash ($ x)
  
    m >>= k   = Squash $ \bx ->
        getSquashed m $ \a ->
        getSquashed (k a) bx

instance Applicative (Squashed c) where
    pure    = return
    liftA2  = liftM2
instance Functor (Squashed c) where
    fmap    = liftM
\end{code}

|Monad|-instance allows to work on the wrapped value, for example
\begin{code}
squashedTree' :: Squashed Monoid (Tree String)
squashedTree' = pure $ Node "x" [ pure "yz", pure "foo" ]

squashedTree :: Squashed Monoid (Tree Int)
squashedTree = do
    x <- squashedTree'
    return (fmap length x)
\end{code}
However, we cannot \emph{extract} the original value, only as much as the
constraint let us:
\begin{code}
-- 6
example_1 :: Int
example_1 = getSum (getSquashed squashedTree (foldMap Sum))

-- [1,2,3]
example_2 :: [Int]
example_2 = getSquashed squashedTree (foldMap pure)
\end{code}

This restriction maybe be useful to enforce correctness, without
relying on the module system!

|Squash c x| is a generalised notion of "free |c| over |x|", e.g. |Monoid| as described
in
\href{http://comonad.com/reader/2015/free-monoids-in-haskell/}{Free Monoids in Haskell}%
\footnote{\url{http://comonad.com/reader/2015/free-monoids-in-haskell/}}.
It should be possible to write |c (Squashed c x)| instances for all (reasonable) |c|. Or actually
|(forall x. c' x => c x) => c (Squashed c' a)| after
\href{https://github.com/ghc-proposals/ghc-proposals/blob/master/proposals/0018-quantified-constraints.rst}{Quantified Constraints -proposal}%
\footnote{\url{https://github.com/ghc-proposals/ghc-proposals/blob/master/proposals/0018-quantified-constraints.rst}}
is implemented. (TODO: amend when we have the extension in released GHC).

\begin{code}
instance Semigroup (Squashed Semigroup x) where
    a <> b = Squash $ \k -> getSquashed a k <> getSquashed b k

instance Monoid (Squashed Monoid x) where
  mempty       = Squash $ \ _  -> mempty
  mappend a b  = Squash $ \k   -> getSquashed a k `mappend` getSquashed b k
\end{code}

As with
\href{http://oleg.fi/gists/posts/2018-05-12-singleton-container.html}{Singleton containers}%
\footnote{\url{http://oleg.fi/gists/posts/2018-05-12-singleton-container.html}},
tell me if you have seen this construction in the wild!

\newpage
Note, that |Squash| doesn't let us turn a thing into something it isn't...
\begin{code}
newtype Squashed1 c f x = Squash1
    { getSquashed1 :: forall g. c g => (forall y. f y -> g y) -> g x }
squash1 :: f x -> Squashed1 c f x
squash1 fx = Squash1 ($ fx)

instance Monad (Squashed1 Monad f) where
    return x  = Squash1 $ \ _ -> return x
    m >>= k   = Squash1 $ \f ->
        getSquashed1 m f >>= \y ->
        getSquashed1 (k y) f

instance Applicative (Squashed1 Monad f) where
    pure    = return
    liftA2  = liftM2
instance Functor (Squashed1 Monad f) where
    fmap    = liftM
\end{code}
... though we can foolishly think so:
\begin{code}
intSet' :: Squashed1 Monad Set Int
intSet' = squash1 $ Set.fromList [1, 2, 3]

intSet :: Squashed1 Monad Set Int
intSet = intSet' >>= \ _ -> return 5

-- [5,5,5]
intList :: [Int]
intList = getSquashed1 intSet Set.toList
\end{code}
So |Squash| let's only forget, not to "remember" anything new.

By the way, this post is genuine Literate Haskell file, using \LaTeX, not Markdown.
If interested on how, check
\href{https://github.com/phadej/gists}{the gists repository}%
\footnote{\url{https://github.com/phadej/gists}}.
I'm weird, as after some point of markup complexity, I actually prefer \LaTeX.
\end{document}
